%% Time-stamp: <2013-05-16 11:18:03 vk>
%%%% === Disclaimer: =======================================================
%% created by
%%
%%      Karl Voit
%%
%% using GNU/Linux, GNU Emacs & LaTeX 2e
%%

%% overrides the default preamble

\documentclass[%
fontsize=\myfontsize,%% size of the main text
paper=\mypapersize,  %% paper format
%parskip=\myparskip,  %% vertical space between paragraphs (instead of indenting first par-line)
DIV=14,           %calc %% page side space%% calculates a good DIV value for type area; 66 characters/line is great
headinclude=true,    %% is header part of margin space or part of page content?
footinclude=false,   %% is footer part of margin space or part of page content?
open=right,          %% "right" or "left": start new chapter on right or left page
appendixprefix=true, %% adds appendix prefix; only for book-classes with \backmatter
bibliography=totoc,  %% adds the bibliography to table of contents (without number)
%liststotoc,        %% adds the list of figures to table of contents
listof=totocnumbered,%% changes list of figures to chapter with numbering
listof=leveldown,    %% cahnges list of figures from chapter to section
draft=\mydraft,      %% if true: included graphics are omitted and black boxes
                     %%          mark overfull boxes in margin space                   
BCOR=\myBCOR,        %% binding correction (depends on how you bind
                     %% the resulting printout.
pointlessnumbers,    %% removes dot after chapter numbering                   
\mylaterality        %% oneside: document is not printed on left and right sides, only right side
                     %% twoside: document is printed on left and right sides                     
]{scrbook}  %% article class of KOMA: "scrartcl", "scrreprt", or "scrbook".
            %% CAUTION: If documentclass will be changed, *many* other things
            %%          change as well like heading structure, ...

% FIXXME: adopting class usage:
% from scrbook -> scrartcl OR scrreport:
% - remove appendixprefix from class options
% - remove \frontmatter \mainmatter \backmatter \appendix from main.tex

% FIXXME: adopting language:
% add or modify language parameter of package »babel« and use language switches described in babel-documentation
 
%\usepackage{ucs}             %% UTF8 as input characters; UCS incompatible to biblatex
\usepackage[utf8]{inputenc} %% UTF8 as input characters

%% used languages; default language is *last* language of options
\usepackage[\mylanguage]{babel}

%% advanced page style using KOMA
%% refer to KOMA documentation for further options
\usepackage{scrpage2}


%%%% !! IMPORTANT THIS PACKAGE NEEDS OPTIONS SET IN THE LATEX EDITOR !!
%% tested with texpad (on osx)
%% -shell-excape
%% PROJECT FILES MUST NOT BE HIDDEN IN SEPARATE FOLDER

\usepackage[xindy]{imakeidx} %% for multiple index and parted alphabetically (Unterteilung mit Buchstaben)
\usepackage[backend=biber, %% using "biber" to compile references (instead of "biblatex")
style=\mybiblatexstyle, %% see biblatex documentation
%%% style=alphabetic, %% see biblatex documentation
%%% dashed=\mybiblatexdashed, %% do *not* replace recurring reference authors with a dash
backref=\mybiblatexbackref, %% create backlings from references to citations
url=false, %% prevents the usage of URLs from a bib file. Activate to use URLs.
natbib=true, %% offering natbib-compatible commands
hyperref=true, %% using hyperref-package references
namefont=smallcaps, %%% Lastname as smallcaps (Kapitälchen)
maxnames=99, %%% lässt alle Autoren und Herausgeber anzeigen!
xref=true, %%% creates a reference from incollection (Artikel) to collection (Sammelbänden) shows both in bibliography 
]{biblatex}  %% remove, if using BibTeX instead of biblatex


\usepackage{chngcntr}
%\counterwithout{footnote}{chapter} %% footnotes are numbered contiously

%%% changes colon to comma in citations (between Author and Titel)
%% Example  Author: Titel => Author, Titel
\renewcommand{\nametitledelim}{\addcomma\space}

%% with multiple authors set firstname before lastname 
\DeclareNameAlias{sortname}{last-first}

%% with multiple autors/publisher(Autoren/Herausgeber) 
%% the names in the bibliography are parted by semicolons, 
%% the last will still use 'and' (und)
\renewcommand{\bibmultinamedelim}{\addsemicolon\space}

%%% changes the parting of the last also to semicolon
\renewcommand{\bibfinalnamedelim}{\addsemicolon\space}

% removes S. / p. befor page numbers in footnotes 
\DeclareFieldFormat{postnote}{#1}

% removes pages numbers in footnotes 
\DeclareFieldFormat{urlprefix}{#1}

% removes S. befor pagenumbers in Bibliography (not used) 
%\DeclareFieldFormat{pages}{#1}

\addbibresource{\mybiblatexfile} %% remove, if using BibTeX instead of biblatex

% Verweise innerhalb des Textes machen:\nameref{Kapitelname}= Verweis auf das Kapitel (Achtung: Das Kapitel vorher als \label{Kapitelname} definieren)\autoref{Kapitelnummer} \pageref{Kapitelname}= verweist auf Seite! Kann auch in Fußnoten eingebunden sein!


% The widely used package to use graphical images within a \LaTeX{} document.
\usepackage[pdftex]{graphicx}

% For additional special characters available by \verb#\ding{}#
\usepackage{pifont}  %% special character for titlepage (Titelseite) \ding{}

% For using if/then/else statements for example in macros
\usepackage{ifthen}  

%% pre-define ifthen-boolean variables:
\newboolean{myaddcolophon}
\newboolean{myaddlistoftodos}

% Using the character for Euro with \verb#\officialeuro{}#
%\usepackage{eurosym}

% This package is required for intelligent spacing after commands
\usepackage{xspace}

%%%%!!!!!!!!!!%%%%%%%
\usepackage{setspace}

% This package defines basic colors. If you want to get rid of colored links and headings
% please change corresponding value in \texttt{main.tex} to \{0,0,0\}.
\usepackage[usenames,dvipsnames]{xcolor}
\definecolor{DispositionColor}{RGB}{\mydispositioncolor} %% used for links and so forth in screen-version

% This package offers strikethrough command \verb+\sout{foobar}+.
\usepackage[normalem]{ulem}

% Create framed, shaded, or differently highlighted regions that can 
% break across pages.  The environments defined are 
\usepackage{framed}

% For example on title pages you might want to have a logo on the upper right corner of
% the first page (only). The package eso-pic is able to place things on absolute
% and relative positions on the whole page.
\usepackage{eso-pic} %%

% This package replaces the built-in definitions for enumerate, itemize and description. 
% With enumitem the user has more control over the layout of those environments.
\usepackage{enumitem} %%

%% option "disable" removes all todonotes output from resulting document
\usepackage[\mytodonotesoptions]{todonotes} 

% For setting correctly typesetted units and nice fractions with \verb+\unit[42]{m}+ and \verb+\unitfrac[100]{km}{h}+.
\usepackage{units}

% uses longtable, tabu for tables
\usepackage{longtable,tabu}

% uses landscape to make pages landscape, or tables on pages
\usepackage{lscape}

% generates random text for showcasing
\usepackage{blindtext}

%%%% End 
%%% Local Variables:
%%% TeX-master: "../main"
%%% mode: latex
%%% mode: auto-fill
%%% mode: flyspell
%%% eval: (ispell-change-dictionary "en_US")
%%% End:
%% vim:foldmethod=expr
%% vim:fde=getline(v\:lnum)=~'^%%%%'?0\:getline(v\:lnum)=~'^%doc.*\ .\\%(sub\\)\\?section{.\\+'?'>1'\:'1':
